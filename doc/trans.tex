\documentclass{article}
\bibliographystyle{unsrt}

% Section
\newcommand{\secref}[1]{Subsection~\ref{sec.#1}}
\newcommand{\seclabel}[1]{\label{sec.#1}}

% Appendix
\newcommand{\appref}[1]{Appendix~\ref{app.#1}}
\newcommand{\applabel}[1]{\label{app.#1}}

% Table
\newcommand{\tabnum}[1]{\ref{tab.#1}}
\newcommand{\tabref}[1]{Table~\tabnum{#1}}
\newcommand{\tablabel}[1]{\label{tab.#1}}

% Equation
\newcommand{\eqnref}[1]{Equation~\ref{Equations.#1}}
\newcommand{\eqnlabel}[1]{\label{Equations.#1}}

% Figure
\newcommand{\figref}[1]{Figure~\ref{Figures.#1}}
\newcommand{\figlabel}[1]{\label{Figures.#1}}

\newcommand{\easyfig}[4]{
\begin{figure}
\includegraphics[#2]{#1}
\caption{ \figlabel{#3} #4}
\end{figure}}

\newcommand{\dotfig}[4]{
\begin{figure}
\input{#1.tex}
\includegraphics[#2]{#1.ps}
\caption{ \figlabel{#3} #4}
\end{figure}}

\newcommand{\pngfig}[2]{\easyfig{#1.png}{}{#1}{#2}}
\newcommand{\pdffig}[2]{\easyfig{#1-fig.pdf}{}{#1}{#2}}

\newcommand{\widepngfig}[2]{\easyfig{#1.png}{width=\textwidth}{#1}{#2}}
\newcommand{\tallpngfig}[2]{\easyfig{#1.png}{height=.8\textheight}{#1}{#2}}
\newcommand{\widepdffig}[2]{\easyfig{#1-fig.pdf}{width=\textwidth}{#1}{#2}}
\newcommand{\tallpdffig}[2]{\easyfig{#1-fig.pdf}{height=.8\textheight}{#1}{#2}}

\newcommand{\sidewayspngfig}[2]{
\begin{sidewaysfigure}
\includegraphics[width=\textwidth]{#1.png}
\caption{\figlabel{#1} #2}
\end{sidewaysfigure}
}

% ToDo
\newcommand{\needfig}[1]{{\bf Need figure: } #1 }
\newcommand{\needfigref}[1]{Figure~??? [#1] }

\newcommand{\needcite}[1]{[CITE #1]}
\newcommand{\todo}[1]{[TODO: #1]}

% Packages

\usepackage[boxed,noend]{algorithm2e}
\usepackage{graphicx}
\usepackage{psfrag}
\usepackage{url}
\usepackage{amsmath}
\usepackage{amssymb}
\usepackage{color} 
\usepackage{ifthen}
\usepackage{rotating}
\newcommand{\hl}[1]{#1}

\newcommand\semiring{K}
\newcommand\srset{\mathbb{K}}
\newcommand\srplus{\oplus}
\newcommand\srtimes{\otimes}
\newcommand\srplusid{\underline{0}}
\newcommand\srtimesid{\underline{1}}
\newcommand\srsum{\bigoplus}
\newcommand\srprod{\bigotimes}

\newcommand\inalph{\Sigma}
\newcommand\outalph{\Omega}
\newcommand\states{Q}
\newcommand\transitions{E}
\newcommand\initstate{i}
\newcommand\finalstate{f}
\newcommand\emptystring{\epsilon}
\newcommand\maybe[1]{#1 \cup \{ \emptystring \}}

\newcommand\state{q}
\newcommand\trans{t}
\newcommand\src{p}
\newcommand\dest{q}
\newcommand\lab{\ell}
\newcommand\inlab{\lab_i}
\newcommand\outlab{\lab_o}
\newcommand\weight{\mathbb{W}}

\newcommand\somealph{\Gamma}
\newcommand\someseq{\gamma}
\newcommand\midlab{\lab_m}

\newcommand\transcomp{\circ}
\newcommand\transconcat{+}

\newcommand\inseq{\sigma}
\newcommand\outseq{\omega}

\newcommand\kleene[1]{{#1}^{\ast}}
\newcommand\inseqs{\kleene{\inalph}}
\newcommand\outseqs{\kleene{\outalph}}
\newcommand\someseqs{\kleene{\somealph}}

\newcommand\tfunc[1]{\mathbb{#1}}

\newcommand\bigo{{\cal O}}

\newcommand\seqof[1]{\bar{#1}}
\newcommand\sympast{v}
\newcommand\symnext{w}
\newcommand\seqpast{\seqof{\sympast}}
\newcommand\seqnext{\seqof{\symnext}}

\newcommand\nucalph{\somealph_{\mbox{\tiny DNA}}}
\newcommand\sym{x}
\newcommand\comp[1]{\tilde{#1}}
\newcommand\revcomp[1]{\comp{#1}}
\newcommand\ntrans[1]{\hat{#1}}
\newcommand\seqlen[1]{|#1|}
\newcommand\otherseq{\lambda}

\newcommand\kmerlen{{\cal K}}
\newcommand\invreplen{L_{\mbox{\tiny invrep}}}

\newcommand\graph{{\cal G}}
\newcommand\vertices{{\cal V}}
\newcommand\edges{{\cal E}}

\newcommand\debruijngraph{\graph_b}
\newcommand\debruijnvertices{\vertices_b}
\newcommand\debruijnedges{\edges_b}

\newcommand\norepgraph{\graph_r}
\newcommand\norepvertices{\vertices_r}
\newcommand\norepedges{\edges_r}

\newcommand\controlgraph{\graph_c}
\newcommand\controlbridges[1]{\vertices_c(#1)}

\newcommand\prequels{{\cal S}}
\newcommand\stepsto{{\cal L}}

\newcommand\ncontrols{N_c}
\newcommand\controlset{{\cal M}}
\newcommand\controlword{M}

\newcommand\initword{\controlword_i}
\newcommand\finalword{\controlword_f}

\newcommand\initvertex[1]{u^{(#1)}_i}
\newcommand\finalvertex[1]{v^{(#1)}_f}

\newcommand\controlmachine{T_c}

\newcommand\bit[1]{#1_2}
\newcommand\trit[1]{#1_3}
\newcommand\quat[1]{#1_4}
\newcommand\controlsym[1]{{#1}_M}
\newcommand\flushsym{F}

\newcommand\figtrans[2]{T_{\ref{Figures.#1}#2}}
\newcommand\solefigtrans[1]{T_{\ref{Figures.#1}}}

\newcommand\transsingle[2]{T_{#1 \to #2}}
\newcommand\transid[1]{\transsingle{#1}{#1}}
\newcommand\transerase[1]{\transsingle{#1}{\epsilon}}

\newcommand\rbit[1]{#1_{2R}}

\newcommand\hammingthreeone{\figtrans{HammingTransducer}{a}}
\newcommand\hammingsevenfour{\figtrans{HammingTransducer}{b}}

\newcommand\transbintoquat{\figtrans{NaiveDNATransducer}{a}}
\newcommand\transquattodna{\figtrans{NaiveDNATransducer}{b}}
\newcommand\transbintodna{\figtrans{NaiveDNATransducer}{c}}

\newcommand\transdivthree{\figtrans{DivisionByThree}{a}}
\newcommand\transternzeroid{\figtrans{DivisionByThree}{b}}
\newcommand\transternsymid{\figtrans{DivisionByThree}{c}}
\newcommand\transternseqid{\figtrans{DivisionByThree}{d}}
\newcommand\transerasezerobits{\figtrans{DivisionByThree}{e}}
\newcommand\transechodivthree{\figtrans{DivisionByThree}{f}}

\newcommand\transmixed{\figtrans{Converter}{a}}
\newcommand\transflush{\figtrans{Converter}{b}}

\newcommand\transechodivtwo{\figtrans{EvenDivision}{a}}
\newcommand\transechodivfour{\figtrans{EvenDivision}{b}}
\newcommand\transechodivmixed{T_{2 \to 234}}

\newcommand\transpartial{\figtrans{PartialObservation}{c}}

\newcommand\nuc[1]{\mbox{#1}}
\newcommand\nuca{\nuc{A}}
\newcommand\nucc{\nuc{C}}
\newcommand\nucg{\nuc{G}}
\newcommand\nuct{\nuc{T}}

\newcommand\contextlen{L}
\newcommand\seqsyms[2]{\sym_{#1} \ldots \sym_{#2}}
\newcommand\seqpastsyms[2]{\sympast_{#1} \ldots \sympast_{#2}}
\newcommand\seqnextsyms[2]{\symnext_{#1} \ldots \symnext_{#2}}
\newcommand\outsym{y}

\newcommand\param[1]{\mathbb{P}_{#1}}

\newcommand\pnogap{\param{n}}
\newcommand\pdelopen{\param{d}}
\newcommand\ptandup{\param{t}}
\newcommand\pfwddup{\param{f}}
\newcommand\prevdup{\param{r}}

\newcommand\pdelext{\param{x}}
\newcommand\pdelend{\param{e}}

\newcommand\pmatch{\param{m}}
\newcommand\ptransition{\param{i}}
\newcommand\ptransversion{\param{v}}

\newcommand\submat{\param{s}}
\newcommand\lendist{\param{l}}

\newcommand\psub{\submat(\sym,\outsym)}
\newcommand\psubnext[1]{\submat(\symnext_{#1},\outsym)}
\newcommand\psubpast[1]{\submat(\sympast_{#1},\outsym)}
\newcommand\psubcomppast[1]{\submat(\sympast_{#1},\comp{\outsym})}
\newcommand\psubcompnext[1]{\submat(\symnext_{#1},\comp{\outsym})}

\newcommand\plen[1]{\lendist(#1)}

\newcommand\pcont{\pnogap^\ast}

\newcommand\block{\alpha}
\newcommand\srcblock{\block}
\newcommand\destblock{\beta}

\newcommand\blockstate[2]{{#1}_{#2}}
\newcommand\lenstate[3]{\blockstate{#1}{#2}^{(#3)}}

\newcommand\symstate[1]{\blockstate{S}{#1}}
\newcommand\delstate[1]{\blockstate{D}{#1}}
\newcommand\tanstate[2]{\lenstate{T}{#1}{#2}}
\newcommand\fwdstate[2]{\lenstate{F}{#1}{#2}}
\newcommand\revstate[2]{\lenstate{R}{#1}{#2}}

\begin{document}

\newcommand\authorstring{
Ian Holmes$^{1,\ast}$ \\
\textbf{1} Department of Bioengineering, University of California, Berkeley, CA, USA \\
$\ast$ E-mail: ihh@berkeley.edu
}

\newcommand\titlestring{Transducer codes for DNA}
\newcommand\shorttitlestring{Transducer codes for DNA}
\markboth{\shorttitlestring}{\shorttitlestring}
\begin{flushleft}
{\Large \textbf{\titlestring} } \\
\authorstring
\end{flushleft}
%\section*{Abstract}
%\paragraph{Keywords:}
%\tableofcontents

\section*{Introduction}

Transducers \cite{MohriPereiraRiley2000,WikipediaTransducers}

In bioinformatics:
protein classification \cite{EskinEtAl2000},
phylogenetics \cite{PatenEtAl2008,WestessonEtAlArxiv2012,WestessonEtAl2012},
cancer informatics \cite{SchwarzEtAl2014}

Arithmetic coding \cite{Mackay2003}

DNA storage \cite{ChurchEtAl2012,GoldmanEtAl2013}.
Codes \cite{YazdiEtAl2015}


\section*{Methods}

\subsection*{Weighted finite-state transducers}

Following \cite{MohriPereiraRiley2000}:
Assume a general semiring
$\semiring=(\srset,\srplus,\srtimes,\srplusid,\srtimesid)$
which for our purposes is typically the probability semiring
$(\Re,+,\times,0,1)$
or the tropical semiring
$(\Re_{+} \cup {\infty},\min,+,\infty,0)$.

A weighted finite-state transducer is defined as a tuple
$T = (\inalph,\outalph,\states,\transitions,\initstate,\finalstate)$
consisting of an input alphabet $\inalph$,
an output alphabet $\outalph$ (both alphabets being finite sets),
a finite set of states $\states$,
a finite set of transitions
$\transitions \subseteq \states \times \maybe{\inalph} \times \maybe{\outalph} \times \srset \times \states$,
an initial state $\initstate \in \states$
and a final state $\finalstate \in \states$.

The transducer $T$ can be thought of as an edge-labelled directed graph
where each state is a vertex
and each transition
$\trans = (\src[\trans],\inlab[\trans],\outlab[\trans],\weight[\trans],\dest[\trans]) \in \transitions$
is an edge from state $\src[\trans]$ to state $\dest[\trans]$
with input label $\inlab[\trans]$,
output label $\outlab[\trans]$
and weight $\weight[\trans]$.

A path in $T$ is a series of transitions that form a path in this graph.
The input sequence and output sequence for a path are the concatenation of (respectively)
the input and output labels of the transitions in the path.
The path weight is the $\srtimes$-product of the transition weights.
A successful path is one that starts in $\initstate$ and ends in $\finalstate$.
The transduction weight for a given input sequence $\inseq \in \inseqs$
and output sequence $\outseq \in \outseqs$
is the $\srplus$-sum of all successful paths
having $\inseq$ as the input sequence
and $\outseq$ as the output sequence.
Thus $T$ provides a mapping
$\tfunc{T}:(\kleene{\inalph} \times \kleene{\outalph}) \to \srset$
from sequence-pairs to weights.
We call this mapping $\tfunc{T}$ the transducer function.
For a given pair of sequences $(\inseq,\outseq)$
and a semiring wherein $\srplus$ and $\srtimes$ are amortized-constant resource operations,
it can be evaluated in time $\bigo(|\inseq| \cdot |\outseq| \cdot |\transitions|)$
and memory $\bigo(|\inseq| \cdot |\outseq| \cdot |\states|)$
by dynamic programming,
analogously to the Forward algorithm in the probabilistic semiring
or the Viterbi algorithm in the tropical semiring
\cite{Durbin98}.

A state $\state \in \states$ has past context $\inseqpast$ if every path from $\initstate$ to $\state$ has an input sequence with suffix $\inseqpast$,
and future context $\inseqnext$ if every path from $\state$ to $\finalstate$ has an input sequence with prefix $\inseqnext$.

\subsection*{Transducer composition}

Given transducers
 $R = (\inalph, \somealph, \states_R, \transitions_R, \initstate_R, \finalstate_R)$ and
 $S = (\somealph, \outalph, \states_S, \transitions_S, \initstate_S, \finalstate_S)$
where $R$'s output alphabet is the same as $S$'s input alphabet,
we can readily find a composite transducer
 $T = R \transcomp S = (\inalph, \outalph, \states_T, \transitions_T, \initstate_T, \finalstate_T)$
such that, if $\tfunc{R}$, $\tfunc{S}$ and $\tfunc{T}$ are the corresponding transducer functions,
then
\[
\forall \inseq \in \inseqs, \outseq \in \outseqs:
\quad
\tfunc{T}(\inseq,\outseq) = \srsum_{\someseq \in \someseqs} \tfunc{R}(\inseq,\someseq) \tfunc{R}(\someseq,\outseq)
\]
that is, $T$ models the feeding of $R$'s output into $S$'s input
(and this intermediate sequence is then summed out---i.e. marginalized, if we are in the probabilistic semiring).

Loosely speaking, we can construct $T$ using the following recipe:
\begin{itemize}
\item Each $T$-state corresponds to a pair of $R$- and $S$-states,
so $\state_T = (\state_R, \state_S)$
and $\states_T \subseteq \states_R \times \states_S$.
\item The initial $T$-state $\initstate_T=(\initstate_R,\initstate_S)$ pairs the initial states of $R$ and $S$.
\item The final $T$-state $\finalstate_T=(\finalstate_R,\finalstate_S)$ pairs the final states of $R$ and $S$.
\item $T$-transitions
$\trans_T = ((\src_R,\src_S),\inlab,\outlab,\weight_T,(\dest_R,\dest_S))$
represent (summaries of sets of) synchronized pairs of $R$-transitions
$\trans_R = (\src_R,\inlab,\midlab,\weight_R,\dest_R)$
and $S$-transitions
$\trans_S = (\src_S,\midlab,\outlab,\weight_S,\dest_S)$
which share the same intermediate label $\midlab$.
The composite transition weight $\weight_T$ is the product $\weight_R \srtimes \weight_S$,
or the sum over such products if there are multiple transition-pairs $(\trans_R,\trans_S)$
consistent with a given $\trans_T$
(which, in the probabilistic semiring, is equivalent to marginalizing out $\midlab$).
\item Some additional manipulation is required to synchronize
transitions involving empty input labels (``insertions''),
empty output labels (``deletions''),
or both (``null transitions'').
For example, we can require that $S$-transitions accepting incoming symbols
all originate from ``ready'' states which have no outgoing insertions.
(If $S$ does not comply with this stipulation then we can easily derive a transducer
with the same transducer function $\tfunc{S}$ that does comply.)
\end{itemize}

Examples of this construction are given in \cite{MohriPereiraRiley2000} and \cite{WestessonEtAlArxiv2012,WestessonEtAl2012}.

\subsection*{De Bruijn graphs}

The $N$-dimension De Bruijn graph over an alphabet $\inalph$ of cardinality $M$
is a directed graph with $M^N$ vertices,
corresponding to all possible $N$-symbol strings
$x_1 x_2 \ldots x_N \in \inalph^N$.
There is an edge $u \to v$ between any two vertices
$u=x_1 x_2 \ldots x_N$ and $v=x_2 x_3 \ldots x_{N+1}$
that overlap by $N-1$ symbols.
Thus, each vertex has $M$ incoming and $M$ outgoing edges \cite{DeBruijn1946,PevznerEtAl2001}.

\subsection*{A transducer for encoding signals in locally non-repetitive DNA}

\subsection*{Allowing for reserved control words}

\subsection*{A transducer that periodically flushes its output}

\subsection*{A transducer that converts a binary sequence into a mixture of binary, trinary and quaternary}



\bibliography{trans}



\end{document}
